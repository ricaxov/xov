\documentclass[a4paper,12pt]{article}
\usepackage[utf8]{inputenc}
\usepackage[brazil]{babel}
\usepackage{amsmath, amssymb}
\usepackage{geometry}
\geometry{a4paper, left=25mm, right=25mm, top=25mm, bottom=25mm}
\usepackage{hyperref}
\usepackage{graphicx}
\usepackage{float}

\title{CMP1170 A01 CG - 2025S1 \\ CG Lista 04}
\author{}
\date{}

\begin{document}
\maketitle
\hrule
\vspace{0.5cm}

\section*{Q1: Equação Matricial de Hermite}

\subsection*{a) Equação Matricial de Hermite}
A forma matricial padrão para uma curva cúbica de Hermite é dada por:
\[
\mathbf{p}(u) = \begin{bmatrix} u^3 & u^2 & u & 1 \end{bmatrix} \; \mathbf{M_H} \; \mathbf{G_H}, \quad u \in [0,1]
\]

\subsection*{b) Matrizes e Vetores Envolvidos}
\begin{itemize}
    \item \textbf{Vetor de Parâmetros:} 
    \[
    \mathbf{U} = \begin{bmatrix} u^3 & u^2 & u & 1 \end{bmatrix}
    \]
    \item \textbf{Matriz de Base de Hermite:}
    \[
    \mathbf{M_H} = 
    \begin{bmatrix}
    2 & -2 & 1 & 1 \\
    -3 & 3 & -2 & -1 \\
    0 & 0 & 1 & 0 \\
    1 & 0 & 0 & 0
    \end{bmatrix}
    \]
    \item \textbf{Vetor de Geometria:}
    \[
    \mathbf{G_H} = 
    \begin{bmatrix}
    P_0 \\
    P_1 \\
    T_0 \\
    T_1
    \end{bmatrix}
    \]
    onde \( P_0 \) e \( P_1 \) são os pontos finais e \( T_0 \) e \( T_1 \) são os vetores tangentes.
\end{itemize}

\subsection*{c) Função de Cada Componente}
\begin{itemize}
    \item \textbf{Vetor \(\mathbf{U}\):} Contém as potências de \( u \) que geram os coeficientes do polinômio cúbico.
    \item \textbf{Matriz \(\mathbf{M_H}\):} Converte as condições geométricas (posição e direção) nos coeficientes do polinômio.
    \item \textbf{Vetor \(\mathbf{G_H}\):} Armazena os pontos de controle e os vetores tangentes, definindo a forma e a orientação da curva.
\end{itemize}

\hrule
\vspace{0.5cm}

\section*{Q2: Equação Matricial de Bézier}

\subsection*{a) Equação Matricial de Bézier}
A equação para uma curva cúbica de Bézier é dada por:
\[
\mathbf{p}(u) = \begin{bmatrix} u^3 & u^2 & u & 1 \end{bmatrix} \; \mathbf{M_B} \; \mathbf{G_B}, \quad u \in [0,1]
\]

\subsection*{b) Matrizes e Vetores Envolvidos}
\begin{itemize}
    \item \textbf{Vetor de Parâmetros:} 
    \[
    \mathbf{U} = \begin{bmatrix} u^3 & u^2 & u & 1 \end{bmatrix}
    \]
    \item \textbf{Matriz de Base de Bézier:}
    \[
    \mathbf{M_B} = 
    \begin{bmatrix}
    -1 & 3 & -3 & 1 \\
    3 & -6 & 3 & 0 \\
    -3 & 3 & 0 & 0 \\
    1 & 0 & 0 & 0
    \end{bmatrix}
    \]
    \item \textbf{Vetor de Geometria:}
    \[
    \mathbf{G_B} =
    \begin{bmatrix}
    P_0 \\
    P_1 \\
    P_2 \\
    P_3
    \end{bmatrix}
    \]
    onde \( P_0, P_1, P_2 \) e \( P_3 \) são os pontos de controle.
\end{itemize}

\subsection*{c) Função de Cada Componente}
\begin{itemize}
    \item \textbf{Vetor \(\mathbf{U}\):} Fornece a base polinomial (com os termos \( u^3, u^2, u, 1 \)) para avaliar a curva.
    \item \textbf{Matriz \(\mathbf{M_B}\):} Define a base de Bézier, determinando a influência de cada ponto de controle na forma da curva.
    \item \textbf{Vetor \(\mathbf{G_B}\):} Contém os pontos de controle que definem a tangência e a curvatura da curva.
\end{itemize}

\hrule
\vspace{0.5cm}

\section*{Q3: Desenho e Interação com uma Curva Cúbica de Bézier em 3D}

\subsection*{a) Desenho da Curva}
\begin{enumerate}
    \item Definir os pontos de controle \( P_0, P_1, P_2, P_3 \) (cada um com coordenadas \((x,y,z)\)).
    \item Para cada valor de \( u \) (variando de 0 a 1, por exemplo, com um incremento pequeno como 0.01), calcular:
    \[
    \mathbf{p}(u) = \begin{bmatrix} u^3 & u^2 & u & 1 \end{bmatrix} \; \mathbf{M_B} \; \mathbf{G_B}
    \]
    \item Plotar os pontos resultantes em um ambiente 3D (utilizando, por exemplo, Python com \texttt{matplotlib} ou bibliotecas gráficas como OpenGL/Three.js).
\end{enumerate}

\subsection*{b) Implementação de Arrasto dos Pontos}
\begin{enumerate}
    \item \textbf{Detecção de Eventos do Mouse:} Detectar cliques próximos aos pontos de controle.
    \item \textbf{Atualização da Posição:} Ao detectar o movimento do mouse com o botão pressionado, atualizar as coordenadas do ponto selecionado.
    \item \textbf{Redesenho da Curva:} Recalcular a curva com os novos pontos de controle e atualizar o desenho em tempo real.
\end{enumerate}

\hrule
\vspace{0.5cm}

\section*{Q4: Rotacionar a Curva em Torno da Reta AB em 3D}

\subsection*{a) Procedimento para Rotação}
Dada a reta definida por:
\[
A(20,30,40) \quad \text{e} \quad B(20,100,40)
\]

\begin{enumerate}
    \item \textbf{Determinar o Eixo de Rotação:}  
    O vetor que define a direção de AB é:
    \[
    \vec{AB} = B - A = (0,70,0)
    \]
    indicando que o eixo é paralelo ao eixo \( y \).
    
    \item \textbf{Translação para a Origem:}  
    Subtrair \( A \) de todos os pontos da curva:
    \[
    \mathbf{p}' = \mathbf{p} - A
    \]
    
    \item \textbf{Aplicar a Rotação:}  
    Utilizar a matriz de rotação em torno do eixo \( y \):
    \[
    R_y(\theta) =
    \begin{bmatrix}
    \cos\theta & 0 & \sin\theta \\
    0 & 1 & 0 \\
    -\sin\theta & 0 & \cos\theta
    \end{bmatrix}
    \]
    e calcular:
    \[
    \mathbf{p}'' = R_y(\theta) \cdot \mathbf{p}'
    \]
    
    \item \textbf{Translação Inversa:}  
    Retornar os pontos para a posição original:
    \[
    \mathbf{p}_{rot} = \mathbf{p}'' + A
    \]
\end{enumerate}

Este procedimento rotaciona a curva em torno da reta definida por \( A \) e \( B \).

\end{document}
